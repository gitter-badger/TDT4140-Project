\documentclass[a4paper]{article}
\usepackage[T1]{fontenc}
\usepackage[utf8]{inputenc}
\usepackage{amsmath}
\usepackage{amssymb}
\usepackage{hyperref}
\usepackage{parskip}
\usepackage{float}
\usepackage{graphicx}
\usepackage{listings}
\usepackage{cleveref}
\usepackage{listings}

\title{Software Engineering TDT4140 - Concept Proposition}
\author{Sindre Hansen \and Vegard Helgesen Hesselberg \and Eirik Rismyhr \and Stian Sørli}
\date{Spring 2017}

\begin{document}
\maketitle
\rule{\linewidth}{0.5mm}

\section{Top five problems of university education}
We interviewed 3 professors, two from NTNU and one external. We studied our interview notes and found the 5 most prominent and important issues. Not all the problems are solvable by a bot, however.

\subsection{Understanding}
One of the hardest problems is to judge to which extent the subject matter is being taken in by the students. It is hard to know where they lie in terms of theoretical and practical knowledge, and where they have holes. This is a hard problem to solve, as it requires a lot of input from students and processing of that input by the teacher or teaching assistants. In subjects where the exercises have to be approved by student assistants (SA), the SAs can formulate an average of the skill level of the students and what points should be repeated in the lecture, and give that to the teacher and Teaching Assistants (TA).

\subsection{Varying levels of difficulty}
Large classes have students with widely different levels of experience and knowledge. Adjusting the difficulty level of the assignments is not easy, because they can’t be too easy for the experienced students or too difficult for the less experienced. A possible solution to this problem is to allow the students to choose tasks from several difficulty levels so that they get tasks that are suitable for them.

\subsection{Facilitating further discussion}
It is difficult to facilitate further discussion among students. Experience indicate that students need to be coerced into discussion. A possible solution to this problem could be to create a bot that creates student groups of 3 to 5 and places them in a chat room with a designated topic. The chat logs can then be sent to the teacher and/or TAs for evaluation.

\subsection{Practical and relevant goals}
Each course should have relevant exercises, so that students can acquire skills they could use when working in a business. One possibility is to assign the students a large project, which is divided into smaller exercises. This might reduce copying of solutions. Grading could be done on the whole project instead of individual exercises. This could replace a traditional written exam. This way of working is more similar to how it's done in the real world.

\subsection{Automation}
In the current system educators either spend a large amount of time creating new exercises or they reuse old ones, making it easy for students to copy old solutions. In technical classes this is a problem that can be solved fairly easily with automation. Exercises can be auto generated around a theme or based on a template exercise where parameters can be tweaked to adjust the difficulty. Automation could also be of great use in grading, particularly if the tasks already are auto generated.

\section{Interviews}
\subsection{Svein Sunde}
Professor, Department of Materials Science and Engineering. His areas of specialty include electrochemistry, water electrolysis and modeling of electrochemical systems. He is a course coordinator in TMT4115 -- General Chemistry and TMT4252 -- Electrochemistry.

\subsection{Frank Lindseth}
Associate Professor at IDI, lecturer in Computer vision (TDT4265) and visual computing fundamentals (TDT4195). Frank has worked both as an associate professor at IDI and as a Senior Researcher at the Medical Technology department of SINTEF since 2014, and has published many articles within the field of visual computing.

\subsection{Tom Heine Nätt} 
Works as a Senior Lecturer at University College in Østfold at their department of information science. He teaches mostly web-related subjects, with a heavy focus on computer security. He has ten years of experience in this position, and has written most of the curriculum he uses for his own classes, but also other publications on computer and web security.

\section{Early idea concept}
Our roBOT, named \textit{Breeze}, will solve, or at least lessen the effects of, the "Facilitating further discussion" problem. We believe that a chat where students discuss a topic is a good solution.

\appendix
\section{Original response data}
\subsection{Svein Sunde}
\begin{enumerate}
    \item Knowing exactly how one should teach in a subject, down to which models to use, formulas and explanations.
    \item Presenting the goals of individual lectures.
    \item Achieving the correct relevance, quality and difficulty of exercises.
    \item Avoid students copying each other's answers. So-called "kok".
    \item Quality of educational area (both lectures and exercises).
\end{enumerate}

\subsection{Frank Lindseth}
\begin{enumerate}
    \item There's a difference between learning/understanding and repeating what's written in the textbook.
    \item Through automation of exercises, exams and correcting, lecturers will be able to manage their time better and spend it on dialogue and other more constructive activities.
    \item "Blackboard lecturing" is dying.
    \item Lecturer should have more of a facilitator and guide for the students instead of spending time lecturing.
    \item Students should practice what they learn, actively work on concrete problem solving.
\end{enumerate}

\subsection{Tom Heine Nätt} 
\begin{enumerate}    
    \item Large skill difference between students, exercises either end up too easy for some or too difficult for others.
    \item The division of subjects at their college often causes some pieces of the curriculum to be cut.
    \item Getting the best use of the 180 study points is a challenge, as informatics is such a wide field, it is difficult to decide whether to focus on basic knowledge or more advanced and specific subjects.
    \item It's not easy to motivate students to comprehend the width of informatics, programming students often ask why they need to understand network protocols, and data system workers wonder why they need to learn programming.
\end{enumerate}
\end{document}