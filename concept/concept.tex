\documentclass{article}
\usepackage[T1]{fontenc}
\usepackage[utf8]{inputenc}
\usepackage{amsmath}
\usepackage{amssymb}
\usepackage{hyperref}
\usepackage{parskip}
\usepackage{float}
\usepackage{graphicx}
\usepackage{listings}
\usepackage{cleveref}
\usepackage{listings}

\title{Software Engineering TDT4140 - Concept Proposition}
\author{Sindre Hansen \and Vegard Helgesen Hesselberg \and Eirik Rismyhr \and Stian Sørli}
\date{Spring 2017}


\begin{document}
\begin{figure}
  \centering
  \includegraphics[width=0.5\textwidth]{images/logontnu_eng}
\end{figure}
\maketitle
\rule{\linewidth}{0.5mm}

\section{Top five problems of university education}
We interviewed 3 professors, two from NTNU and one external. We studied our interview notes and found the 5 most prominent and important issues. Not all the problems are solvable by a bot, however.

\subsection{Understanding}
One of the hardest problems is to judge to which extent the subject matter is being taken in by the students. It is hard to know where they lie in terms of theoretical and practical knowledge, and where they have holes. This is a hard problem to solve, as it requires a lot of input from students and processing of that input by the teacher or teaching assistants. In subjects where the exercises have to be approved by student assistants (SA), the SAs can formulate an average of the skill level of the students and what points should be repeated in the lecture, and give that to the teacher and Teaching Assistants (TA).

\subsection{Varying levels of difficulty}
Large classes have students with widely different levels of experience and knowledge. Adjusting the difficulty level of the assignments is not easy, because they can’t be too easy for the experienced students or too difficult for the less experienced. A possible solution to this problem is to allow the students to choose tasks from several difficulty levels so that they get tasks that are suitable for them.

\subsection{Facilitating further discussion}
It is difficult to facilitate further discussion among students. Experience indicate that students need to be coerced into discussion. A possible solution to this problem could be to create a bot that creates student groups of 3 to 5 and places them in a chat room with a designated topic. The chat logs can then be sent to the teacher and/or TAs for evaluation.

\subsection{Practical and relevant goals}
Each course should have relevant exercises, so that students can acquire skills they could use when working in a business. One possibility is to assign the students a large project, which is divided into smaller exercises. This might reduce copying of solutions. Grading could be done on the whole project instead of individual exercises. This could replace a traditional written exam. This way of working is more similar to how it's done in the real world.

\subsection{Automation}
In the current system educators either spend a large amount of time creating new exercises or they reuse old ones, making it easy for students to copy old solutions. In technical classes this is a problem that can be solved fairly easily with automation. Exercises can be auto generated around a theme or based on a template exercise where parameters can be tweaked to adjust the difficulty. Automation could also be of great use in grading, particularly if the tasks already are auto generated.

\section{Interviews}
\subsection{Svein Sunde}
Professor, Department of Materials Science and Engineering. His areas of specialty include electrochemistry, water electrolysis and modeling of electrochemical systems. He is a course coordinator in TMT4115 -- General Chemistry and TMT4252 -- Electrochemistry.

\subsection{Frank Lindseth}
Associate Professor at IDI, lecturer in Computer vision (TDT4265) and visual computing fundamentals (TDT4195). Frank has worked both as an associate professor at IDI and as a Senior Researcher at the Medical Technology department of SINTEF since 2014, and has published many articles within the field of visual computing.

\subsection{Tom Heine Nätt} 
Works as a Senior Lecturer at University College in Østfold at their department of information science. He teaches mostly web-related subjects, with a heavy focus on computer security. He has ten years of experience in this position, and has written most of the curriculum he uses for his own classes, but also other publications on computer and web security.

\section{Early idea concept}
Our roBOT, named \textit{Breeze}, will solve, or at least lessen the effects of, the "Facilitating further discussion" problem. We believe that a chat where students discuss a topic is a good solution.

\appendix

\section{Product backlog}
\begin{tabular}{ c | p{0.5\textwidth} | c | c}
     ID & Story & Estimate & Priority \\ \hline
     T1 & As a user, I want to enter my work hours so I can make sure I get paid on time & 5 & 1 \\
\end{tabular}

\newpage
\section{Activity plan}
\begin{adjustbox}{center}
\begin{tabular}{ p{0.1\paperwidth} | p{0.1\paperwidth} | p{0.4\textwidth} | p{0.1\paperwidth} | p{0.1\paperwidth} }
    Release due date & Story\# or other tasks & Description \newline \textcolor{red}{Remarks} & Estimated resource use & Actual resource use \\ \hline
    27.4.2017 & Project management and related & Initial project plan, wrap-ups \textcolor{red}{Project management, wrap-ups, meetings, coaching} & 11 h & 11 h \\ \cline{2-5}
     & Planning day & lol & 35 h & 37 h \\ 
     
\end{tabular}
\end{adjustbox}

\newpage
\section{Risk assessment plan}
\begin{adjustbox}{center}
\begin{tabular}{ p{0.1\paperwidth} | p{0.25\paperwidth} | p{0.3\paperwidth} }
    Risk & Means to prevent & Action and responsible \\ \hline
    Schedule slips and project cancellation
    & Controlled through daily progress monitoring and short release cycles.
    & When schedule slips \textbf{customer/management} is requested to drop least needed stories/tasks or grant more resources. \textbf{Project manager} is responsible to keep management informed about project status. \\ \hline
    
    Business change, cost of changes
    & New stories accepted on each release’s planning day. Automatic unit tests (regression), no malfunctioning sources allowed in version system.
    & When stories need changing \textbf{customer/management} informs \textbf{project manager} who decides whether requested changes can be made into on-going release, in other case \textbf{customer} must choose stories to drop or grant needed resources. \textbf{Programmers} are responsible to update tests and check-in only test-passing source. \\ \hline
    
\end{tabular}
\end{adjustbox}
\end{document}